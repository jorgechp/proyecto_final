\documentclass{article}
\usepackage{cite}
\usepackage{url}
\usepackage[utf8]{inputenc}

\title{La reducción del número de piratas como causa del calentamiento global}
\author{Bobby Henderson}
\date{October 2019}

\begin{document}

\maketitle

\section{Resumen}

\begin{itemize}
    \item Url del Repositorio: \url{https://github.com/jorgechp/proyecto_final}
\end{itemize}

Se denomina calentamiento global al aumento de la temperatura en el planeta Tierra así como a sus efectos. Si bien el calentamiento global es un fenómeno observado desde hace varias décadas, actualmente se ha convertido en un foco mediático así como en objeto de estudio desde diferentes campos de la ciencia. En este artículo, analizamos el fenómeno del descenso global del número de piratas y planteamos su inevitable relación con respecto al calentamiento global. Tras realizar diferentes pruebas de carácter empírico, encontramos que el número de piratas guarda una estrecha correlación y, por ende, causalidad, con respecto al aumento global de la temperatura en el planeta y con la victoria de Donald Trump como candidato a la presidencia de los Estados Unidos de América. 

\begin{itemize}
	\item Palabras clave: Piratas, calentamiento global, Donald Trump, mayonesa, temperatura.
\end{itemize}

\section{Introducción}

En los dos últimos siglos, nuestro planeta ha experimentado un incremento gradual de la temperatura media, la observación de este fenómeno, así como de sus causas, se denomina Calentamiento Global  \cite{mann_selin_2019, rahayuglobal}. El incremento de temperatura global se ha ido acelerando paulatinamente, siendo en la última década, de aproximadamente 0.93 grados \cite{alen}. 

\bibliography{bibliography}{}
\bibliographystyle{plain}

\end{document}
