\documentclass{article}
\usepackage{cite}
\usepackage{url}
\usepackage[utf8]{inputenc}

\title{La reducción del número de piratas como causa del calentamiento global}
\author{Bobby Henderson}
\date{October 2019}

\begin{document}

\maketitle

\section{Resumen}

\begin{itemize}
    \item Url del Repositorio: \url{https://github.com/jorgechp/proyecto_final}
\end{itemize}

Se denomina calentamiento global al aumento de la temperatura en el planeta Tierra así como a sus efectos. Si bien el calentamiento global es un fenómeno observado desde hace varias décadas, actualmente se ha convertido en un foco mediático así como en objeto de estudio desde diferentes campos de la ciencia. En este artículo, analizamos el fenómeno del descenso global del número de piratas y planteamos su inevitable relación con respecto al calentamiento global. Tras realizar diferentes pruebas de carácter empírico, encontramos que el número de piratas guarda una estrecha correlación y, por ende, causalidad, con respecto al aumento global de la temperatura en el planeta y con la victoria de Donald Trump como candidato a la presidencia de los Estados Unidos de América. 

\begin{itemize}
	\item Palabras clave: Piratas, calentamiento global, Donald Trump, mayonesa, temperatura.
\end{itemize}

\section{Introducción}
\subsection{Cambio climático}

En los dos últimos siglos, nuestro planeta ha experimentado un incremento gradual de la temperatura media, la observación de este fenómeno, así como de sus causas, se denomina Calentamiento Global  \cite{mann_selin_2019, rahayuglobal}. El incremento de temperatura global se ha ido acelerando paulatinamente, siendo en la última década, de aproximadamente 0.93 grados \cite{alen}. 

A nivel social, existe una concienciación cada vez mayor sobre el cambio climático y sus efectos. No obstante, la opinión pública tiende a asumir conceptos erróneos sobre el Cambio climático. Por ejemplo, existen confusión a la hora de distinguir el agujero de la capa de ozono con el efecto invernadero, así como a utilizar indistintamente \emph{clima} y \emph{tiempo}\cite{bostrom}. Si bien es una posición considerada minoritaria, existen voces que niegan el cambio climático, especialmente en el ámbito de empresas que tienen intereses financieros en segmentos de la economía que tienen una incidencia medioambiental muy elevada \cite{astroturf}.

La comunidad académica se centra en los últimos años en el estudio del cambio climático y sus causas. En los últimos años, han cobrado relevancia tesis que identifican las emisiones de C02 como un factor significativo para explicar el fenómeno del aumento de la temperatura global \cite{whitmarsh2011}. En cualquier caso, la hipótesis de que el la acción humana tiene una gran repercusión en el medio ambiente hoy en día se encuentra, generalmente, aceptada.

\subsection{Piratería}
Se entiende por piratería al acto de cometer delitos de pillaje o violencia criminal contra barcos o zonas costeras \cite{pennell_2001}. Los primeros documentos que describen casos de piratería podemos encontrarlos en el siglo XIV cuando un grupo de navegantes, conocidos como los \emph{Pueblos del mar} emprendieron ataques contra diferentes civilizaciones repartidas por el Mar Mediterráneo y el Mar Egéo y en zonas de importancia estratégica como el Estrecho de Gibraltar. No obstante, aunque sin esta denominación, la piratería ya era común en el Siglo II antes de Cristo \cite{lane2015}.

Los piratas tuvieron una importancia muy relevante a partir del descubrimiento de américa, donde las naciones occidentales experimentaron un crecimiento muy importante de sus flotas y se establecieron nuevas rutas comerciales por los Océanos Atlántico y Pacífico. Es también esta época cuando la relación entre piratería y estados gana complejidad, llegando los primeros a realizar pactos con los segundos para defender sus intereses o para boicotear comercial a sus adversarios. Especialmente relevante es el caso de Inglaterra, quien estableció fuertes relaciones con piratas para atacar intereses comerciales de la Corona de Castilla así como de otras potencias rivales \cite{hebb2016}.




\bibliography{bibliography}{}
\bibliographystyle{plain}

\end{document}
